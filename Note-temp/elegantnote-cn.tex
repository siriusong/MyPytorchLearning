%!TEX program = xelatex
\PassOptionsToPackage{prologue,dvipsnames}{xcolor}
\documentclass[cn,hazy,blue,14pt,screen]{elegantnote}
\title{Pytorch Note: 一份(不太)简短的笔记}

\author{Song Chao}
%\institute{Elegant\LaTeX{} Program}

\version{1.0}
\date{\zhtoday}

\usepackage{array}
\usepackage{listings}
% 在导言区进行样式设置
\lstset{
    language=Python, % 设置语言
 basicstyle=\ttfamily, % 设置字体族
 breaklines=true, % 自动换行
 keywordstyle=\bfseries\color{NavyBlue}, % 设置关键字为粗体,颜色为 NavyBlue
 morekeywords={}, % 设置更多的关键字,用逗号分隔
 emph={self}, % 指定强调词,如果有多个,用逗号隔开
    emphstyle=\bfseries\color{Rhodamine}, % 强调词样式设置
    commentstyle=\itshape\color{black!50!white}, % 设置注释样式,斜体,浅灰色
    stringstyle=\bfseries\color{PineGreen!90!black}, % 设置字符串样式
    columns=flexible,
    numbers=left, % 显示行号在左边
    numbersep=2em, % 设置行号的具体位置
    numberstyle=\footnotesize, % 缩小行号
    frame=single, % 边框
    framesep=1em % 设置代码与边框的距离
}
\begin{document}

\maketitle

\centerline{
  \includegraphics[width=0.2\textwidth]{logo-cute.jpg}
}

\section{Pytorch快速入门}

本章主要介绍张量、自动微分、torch.nn模块的卷积池化等操作,以及数据预处理等相关的模块。

\subsection{张量}

张量一共有八种,但是默认的数据类型是32位浮点型(torch.FloatTensor),可以通过\\torch.set\_default\_tensor\_type()函数设置默认的数据类型,但是该函数只支持设置浮点型的数据类型。

\subsubsection{张量的生成}

在程序中使用torch.tensor()函数生成一个张量,然后使用.dtype方法获取其数据类型,可以使用torch.get\_default\_dtype()函数获得默认的数据类型。

通过torch.tensor()函数可以将Python的列表转化为张量,张量的维度使用.shape查看,也可以使用.size()方法计算张量的形状大小,使用.numel()方法计算张量中包含元素的数量。

在使用torch.tensor()函数时,使用参数dtype来指定张量的数据类型,使用requires\_grad来指定张量是否需要计算梯度\footnote{只有浮点类型的张量才允许计算梯度}。

针对已经生成的张量可以使用torch.**\_like()系列函数生成与指定张量维度相同、性质相似的张量,如使用torch.ones\_like(D)生成与D维度相同的全1张量,使用torch.zeros\_like()生成全0张量,使用torch.rand\_like()生成随机张量。针对一个创建好的张量D,使用D.new\_**()系列函数创建新的张量,如使用D.new\_tensor()将列表转化为张量。还有一些函数可以得到新的张量\footnote{D起的作用就是创建的张量和D的数据类型一致}
\begin{itemize}
  \item D.new\_full((3,3),fill\_value=1):$3\times 3$使用1填充的张量;
  \item D.new\_zeros((3,3)): $3\times 3$的全0张量;
  \item D.new\_empty((3,3)): $3\times 3$的空张量;
  \item D.new\_ones((3,3)): $3\times 3$的全1张量。
\end{itemize}

张量和NumPy数组可以相互转换。将NumPy数组转化为Pytorch张量,可以使用torch.as\_tensor()和torch.from\_numpy(),但是需要注意转换成的NumPy数组默认是64位浮点型数据。对于张量,使用torch.numpy()即可转化为NumPy数组。

可以通过相关随机数来生成张量,并且可以指定生成随机数的分布函数等,在生成随机数之前,可以使用torch.manual\_seed(),指定生成随机数的种子,保证生成随机数是可重复出现的。如使用torch.normal()生成服从高斯的随机数,在该函数中,通过mean指定随机数的均值,std参数指定标准差,如果这两个参数只有一个元素则只生成一个随机数,如果有多个值,可以生成多个随机数。也可以使用torch.rand()函数,在区间[0,1]上生成服从均匀分布的张量。使用torch.randn()函数则可生成服从标准正态分布的随机数张量。使用torch.randperm(n)函数,可以将$0\sim n$(包含0,不包含n)之间的整数进行随机排序后输出。

在Pytorch中可以使用torch.arange()和torch.linspace()来生成张量,前者的参数start指定开始,end指定结束,step指定步长;后者是在范围内生成固定数量的等间隔张量;torch.logspace()则可生成以对数为间隔的张量。

\subsubsection{张量操作}

生成张量后,有时需要改变张量的形状、获取或改变张量中的元素、将张量进行拼接和拆分等。

A.reshape()可以将张量A设置为想要的形状大小,或者直接通过torch.reshape()函数改变输入张量的形状,参数input为需要改变的tensor,shape为想要的形状。改变张量的形状可以使用$tensor.resize_()$,针对输入的形状大小对张量形状进行修改。还提供了$A.resize_as_(B)$,可以将张量A的形状尺寸设置为和B一样的形状。

torch.unsqueeze()可以在张量的指定维度插入新的维度得到维度提升的张量,而torch.squeeze()可以移除指定或者所有维度为1的维度,从而得到减小的新张量。

可以使用.expand()对张量的维度进行扩展,而$A.expand_as(C)$方法会将张量A根据张量C的形状大小进行拓展,得到新的张量。使用张量的.repeat()方法可以将张量堪称一个整体,然后根据指定的形状进行重复填充,得到新的张量。

从张量中利用切片和索引提取元素的方法,和在numpy中的使用方法是一致的。也可以按需将索引设置为相应的bool值,然后提取真条件下的内容。

torch.tril()可以获取张量下三角部分的内容,而将上三角的元素设置为0;torch.triu()则相反;\\torch.diag()可以获取矩阵张量的对角线元素,或者提供一个向量生成一个矩阵张量。上述三个函数可以通过diagonal参数来控制所要考虑的对角线。torch.diag()提供对角线元素,来生成对角矩阵。

Pytorch中提供了将多个张量拼接为1个张量,或者将一个张量拆分为几个张量的函数,其中torch.cat()将多个张量在指定的维度进行拼接,得到新的张量。torch.stack()可以将多个张量按照指定的维度进行拼接。torch.chunk()可以将张量分割为特定数量的块;torch.split()在将张量分割为特定数量的块时,可以指定每个块的大小。

\subsubsection{张量计算}

主要包括张量之间的大小比较,基本运算,与统计相关的运算,如排序、最大值及其位置。

针对张量之间的元素比较大小,主要有以下几个
\begin{itemize}
  \item torch.allclose():比较两个元素是否接近,公式为$|A-B|\le atol+rtol \times |B|$;
  \item torch.eq():逐元素比较是否相等;
  \item torch.equal():判断两个张量是否具有相同的形状和元素;
  \item torch.ge():逐元素比较大于等于;
  \item torch.gt():逐元素比较大于;
  \item torch.le():逐元素比较小于等于;
  \item torch.lt():逐元素比较小于;
  \item torch.ne():逐元素比较不等;
  \item torch.isnan():判断是否为缺失值;
\end{itemize}

张量的基本运算,一种为逐元素之间的运算,如加减乘除、幂运算、平方根、对数、数据裁剪等,一种为矩阵之间的运算,如矩阵相乘、矩阵的转置、矩阵的迹等。

计算张量的幂可以用torch.pow(),或者**符号。计算指数可以使用torch.exp();对数为torch.log();开方为torch.sqrt();平方根的倒数为torch.rsqrt()。针对数据的裁剪,有根据最大值裁剪torch.clamp\_max();有根据最小值裁剪torch.clamp\_min();还有根据范围裁剪torch.clamp()。

矩阵运算中,有torch.t()为转置;torch.matmul()输出两个矩阵的乘积;torch.inverse()为矩阵的逆;torch.trace()为矩阵的迹

还有一些基础的统计计算功能,torch.max()计算最大值;torch.argmax()输出最大值所在的位置;torch.min()和torch.argmin()也类似。torch.sort()可以对一维张量进行排序,或者对高维张量在指定的维度进行排序,在输出排序结果的同时,还会输出对应的值在原始位置的索引。torch.topk()根据指定的k值,计算出张量中取值大小为第k大的数值与数值所在的位置;torch.kthvalue()根据指定的k值,计算出张量中取值大小为第k小的数值与数值所在的位置。

还有一些基础函数如下所示:
\begin{itemize}
  \item torch.mean():根据指定的维度计算均值
  \item torch.sum():根据指定的维度求和
  \item torch.cumsum():根据指定的维度计算累加和
  \item torch.median():根据指定的维度计算中位数
  \item torch.cumprod():根据指定的维度计算累乘积
  \item torch.std():计算标准差
\end{itemize}

\subsubsection{Pytorch中的自动微分}

在torch中的torch.autograd模块,提供了实现任意标量值函数自动求导的类和函数,针对一个张量只需要设置参数$requires_grad=True$,通过相关计算即可输出其在传播过程中的梯度信息。在Pytorch中生成一个矩阵张量x,并且y=sum($x^2+2x+1$),计算出y在x上的导数,程序如下。

也可以通过y.backward()来计算y在x的每个元素上的导数。

% lstlisting环境
\begin{lstlisting}
  import random
  import collections
  Card = collections.namedtuple('Card', ['rank', 'suit'])
  # 一个叫做 FrenchDesk 的类。a class named FrenchDesk.
  class FrenchDesk:
      ranks = [str(n) for n in range(2, 11)] + list('JQKA')
      suits = 'spades diamonds clubs hearts'.split()
      
      def __init__(self):
          self._cards = [Card(rank, suit) for rank in self.ranks for suit in self.suits]
          
      def __len__(self):
          return len(self._cards)
          
      def __getitem__(self, position):
          return self._cards[position]
  desk = FrenchDesk()
\end{lstlisting}


%\printbibliography[heading=bibintoc, title=\ebibname]

\end{document}
